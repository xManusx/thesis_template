\usepackage{morewrites} 
\usepackage[autostyle=true]{csquotes}
%For comment environment
\usepackage{verbatim}

%For acronyms
%\usepackage[smaller,printonlyused]{acronym}

\usepackage[inline]{enumitem}
\usepackage{longtable}
\usepackage{expl3} % for sorting acros

%Omit final dot from each def.
\def\eg{e.\,g.}
\def\ie{i.\,e.}
\newcommand{\Set}[1]{%
\{#1\}%
}


\usepackage[binary-units=true]{siunitx}
\sisetup{group-separator = {,}}

\usepackage{eurosym}
\usepackage{wasysym} %for circle symbols


\usepackage{array}% for extended column definitions
\renewcommand\arraystretch{1.5}

\usepackage{relsize} %for resizing acros

\usepackage[
	sort=true
	,single=true,
	,list-name={List of Abbreviations}
	%,list-style=longtable
	%,list-style=description
	,only-used=false
]
{acro}

\acsetup{
	%list-style=tabular
	list-style=longtable
	,list-caps=true
	,short-format={\scshape\small}
	,extra-style=plain
	%,first-style=reversed
	,list-short-format ={\bfseries} % Abkürzungen in fetter Serifenschrift im Verzeichnis 
	%,first-long-format = {\itshape} % Erste Lange Ausführung der Abkürzung kursiv gestellt
}

\DeclareAcroListStyle{tabular}{table}{
	table = longtable,
	% 4 columns: acronym, description, extra information, page number
	table-spec = >{\bfseries}p{0.12\linewidth}p{0.4\linewidth}p{0.36\linewidth}l
}
\DeclareAcroListStyle{extra-tabular}{extra-table}{
  table = longtable,
  % 4 columns: acronym, description, extra information, page number
	table-spec = >{\bfseries}p{0.12\linewidth}p{0.75\linewidth}p{0.01\linewidth}l
}

%\acsetup{list-long-format=\capitalisewords}


%\newlist{acronyms}{description}{1}
%\newcommand*\addcolon[1]{#1:}
%\setlist[acronyms]{
	%labelwidth = 4.5em,
	%leftmargin = 3.5em,
	%noitemsep,
	%itemindent = 0pt,
%font=\addcolon}
%\DeclareAcroListStyle{mystyle}{list}{ list = acronyms }
%\acsetup{ list-style = mystyle }

\usepackage{mfirstuc}% provides \capitalisewords

\setlength{\marginparwidth}{2cm}
\usepackage[
	obeyDraft
	,textwidth=0.9\marginparwidth
	,textsize=footnotesize
	,bordercolor=red
	,linecolor=red
	,backgroundcolor=white
	,shadow
	,figwidth=0.75\linewidth
]{todonotes}

\usepackage{ifdraft}


%\usepackage[tracking=true,draft=false]{microtype}
\usepackage[draft=false,activate={true,nocompatibility},final,tracking=true,kerning=true,spacing=true,factor=1100,stretch=10,shrink=10]{microtype}
\DeclareMicrotypeSet*[tracking]{my}% 
	{ font = */*/*/sc/* }% 
\SetTracking{ encoding = *, shape = sc }{ 45 }% Hier wird festgelegt,
            % dass alle Passagen in Kapitälchen automatisch leicht
            % gesperrt werden. Das Paket soul, das ich früher empfohlen
            % habe ist damit für diese Zwecke nicht mehr nötig.
            %
% activate={true,nocompatibility} - activate protrusion and expansion
% final - enable microtype; use "draft" to disable
% tracking=true, kerning=true, spacing=true - activate these techniques
% factor=1100 - add 10% to the protrusion amount (default is 1000)
% stretch=10, shrink=10 - reduce stretchability/shrinkability (default is 20/20)

%\SetProtrusion{encoding={*},family={bch},series={*},size={6,7}}
%{1={ ,750},2={ ,500},3={ ,500},4={ ,500},5={ ,500},
%6={ ,500},7={ ,600},8={ ,500},9={ ,500},0={ ,500}}

%\SetExtraKerning[unit=space]
%{encoding={*}, family={bch}, series={*}, size={footnotesize,small,normalsize}}
%{\textendash={400,400}, % en-dash, add more space around it
	%"28={ ,150}, % left bracket, add space from right
	%"29={150, }, % right bracket, add space from left
	%\textquotedblleft={ ,150}, % left quotation mark, space from right
%\textquotedblright={150, }} % right quotation mark, space from left

\usepackage[all]{nowidow}

\newcommand{\myinclude}[1]{%
	\fancyhead[RE]{\leftmark}%
	\include{#1}%
	\fancyhead[RE]{\rightmark}%
}

\def\ra{$\rightarrow$}
\def\Ra{$\Rightarrow$}


\usepackage{listings}
\lstset{
	breaklines = true,
	breakatwhitespace = true,
	basicstyle = \ttfamily
}

\lstMakeShortInline[columns=fixed]|

\usepackage{tabularx}

\usepackage{booktabs}

%Tweak that tilde!
\newcommand{\mysim}{%
	{\raise.17ex\hbox{$\scriptstyle\sim$}}%
}

%\usepackage{paralist} %For inline enums
\usepackage[inline]{enumitem}

\input{include/attackdefinitions}

\usepackage{pifont} %for checkmarks and crosses
\newcommand{\cmark}{\ding{51}}%
\newcommand{\xmark}{\ding{55}}%


\newcommand{\epigraph}[2]{%
	\begin{displayquote}
		{\small%
		#1\\
		\phantom{blabla} \hfill#2}
	\end{displayquote}
}
