%%%%%%%%%%%%%%%%%%%%%%%%%%%%%%%%%%%%%%%%%%%%%%%%%%%%%%%%%%%%%%%%%%%%%%%%%%%%%%%
%
% Abstract
% 
%%%%%%%%%%%%%%%%%%%%%%%%%%%%%%%%%%%%%%%%%%%%%%%%%%%%%%%%%%%%%%%%%%%%%%%%%%%%%%%

% Pseudo chapter
\chapter*{\ }
%\thispagestyle{plain}

%\pagebreak

\begin{otherlanguage}{ngerman}
	%\vspace*{0.10\textheight}
	\begin{center}
		\begin{large}
			\textbf{Zusammenfassung}
		\end{large}
	\end{center}
	\vspace{0.75em}

	Die Vision des  \ac{iot} wurde in den letzten Jahren sowohl von der Forschung im IT Bereich, als auch von industriellen Entwicklungen und Produkten geprägt.
	Das neue \ac{iot}-Paradigma propagiert die Vernetzung aller Arten von Geräten -- Kaffeemaschinen, Türschlössen, Wetterstationen, verschiedenen Sensoren, usw. -- und dem Internet.
	Es wird erwartet, dass die Zahl der \ac{iot}-Geräte in den nächsten Jahren massiv zunehmen wird.
	Dieser allgegenwärtige und alles durchdringende Informationsaustausch bedarf starker Anforderungen an die Informationssicherheit.
	Sicherheitsmängel könnten zu schwerwiegenden Beeinträchtigungen der Privatsphäre der Nutzer oder gar zu (physisch) gefährlichen Situationen führen.
	Einer der zahllosen drahtlosen Netzwerkstandards für das \ac{iot} ist \ac{nb}.
	\ac{nb} basiert auf dem \enquote{traditionellen} \ac{lte} Mobilfunkstandard und ist ein so  genanntes
	\ac{lpwan}.
	Diese Klasse von Funkstandards konzentriert sich auf kostengünstige Geräte mit geringer Leistung, aber großer Reichweite.
	Als mögliches Einsatzgebiet werden häufig die Vision der \enquote{Smart City} und ähnliche Konzepte angegeben.
	Als Sicherheitsmechanismen kommen die seit mehreren Jahren bewährten Verfahren von \ac{lte} zum Einsatz.
	Deshalb wird \ac{nb} häufig als eine \enquote{sichere} \ac{lpwan} Lösung bezeichnet.
	Es kann jedoch davon ausgegangen werden, dass nicht nur Sicherheitsfeatures, sondern auch Sicherheitslücken und Schwachstellen von \ac{lte} übernommen wurden.
	Da in der Vergangenheit bereits öfters Schwachstellen und Angriffe gegen \ac{lte} entdeckt wurden, ist es notwendig, auch die Sicherheit von \ac{nb} zu untersuchen.
	Diese Arbeit beschreibt die Architekturen von \ac{lte}  und \ac{nb}, sowie bereits entdeckte Schwachstellen von \ac{lte} um deren Anwendbarkeit auf \ac{nb} auf einer theoretischen Ebene zu untersuchen.
	Es wird gezeigt, dass die meisten Schwachstellen tatsächlich in beiden Standards vorkommen.
	Obwohl diese als für einige \ac{iot} Beispielanwendungen unkritisch eingestuft wurden,  können die Ergebnisse als Beispiel für zukünftige \ac{nb}-Sicherheitsforschung dienen.
	Diese wird in Zukunft wohl noch dringender benötigt werden, falls die Wachstumsprognosen der \ac{iot}-Branche zutreffen sollten und das \ac{iot} noch allgegenwärtiger werden wollte.

	%\vspace*{\fill}
\end{otherlanguage}
\acresetall

\cleardoublepage
%\thispagestyle{plain}
\chapter*{\ }
\acresetall

%\vspace*{0.10\textheight}
\begin{center}
	\begin{large}
		\textbf{Abstract}
	\end{large}
\end{center}
\vspace{0.75em}


In recent years, the vision of the \ac{iot} has been shaped both by Computer Science research as well as by industrial developments and products.
This new paradigm propagates the connectivity of all kinds of devices -- coffee machines, door locks, weather stations, various sensors, etc. --  and the internet.
It is expected that there will be a huge increase in the number of \ac{iot} devices within the next few years.
This ubiquitous and pervasive information exchange mandates a strong need for security.
A lack thereof could cause serious disruptions to the users' privacy and safety.
One of the various wireless networking standards for the \ac{iot} is \ac{nb}, which is built on top of the \enquote{traditional} \ac{lte} cellular standard.
\ac{nb}  is one of several so called \ac{lpwan} standards and focus on low-cost and low-power but long range use cases.
Examples for this use case are commonly found in the vision of the \enquote{Smart City} and similar concepts.
Since the security mechanisms of \ac{nb} are shared with \ac{lte}, which has been in use for several years, it is claimed that \ac{nb} is a secure \ac{lpwan} solution.
However, it can be assumed that not only security features but also security flaws and vulnerabilities are shared between \ac{nb} and \ac{lte}.
As several weaknesses and attacks against \ac{lte} have been discovered, the security of \ac{nb} needs thorough investigation.
This thesis examines both \ac{lte} and \ac{nb} as well as already discovered attacks on \ac{lte} and investigates their applicability to \ac{nb} on a theoretical level.
It has been shown that most vulnerabilities are in fact  shared between both networking standards.
Although these vulnerabilities have been identified as being mostly non-critical for a number of exemplary applications, the results can serve as an example of \ac{nb} security research, that will be urgently needed in the future, where even more devices will be interconnected.
\acresetall

%\begin{itemize}
%\item
%Identify purpose:
%\begin{itemize}
%\item
%Why did I decide to study this topic?

%\Ac{iot} will be important topic in the future, expected growth is huge, vision behind it is fascinating
%\item
%How did I conduct my research?

%Examine both \ac{lte} and \ac{nb}, examine attacks on \ac{lte}, investigate their applicability to \ac{nb} on a theoretical level
%\item
%What did I find?

%Some attacks are based on features that are not supported in \ac{nb} and consequentially are not possible in \ac{nb}.
%Most attacks, however, are applicable to the new standard.
%\item
%Why is this research and findings important?

%Although the presented vulnerabilities have been shown to be mostly non-critical for the example applications, these results can serve as an example of \ac{nb} security research, that will be urgently needed in the future.
%As it is to be assumed that there will be a multitude of \ac{nb} applications in the future and the discovery of more attacks on \ac{lte}, respectively  \ac{nb}, can be expected, security considerations and research need to be a vital part of \ac{iot} application development.
%Thus, 
%\item
%Why should someone read the thesis entirely?

%It gives an overview of relevant parts of both \ac{lte} and \ac{nb}, necessary to understand the vulnerabilities in both standards and examine their impact on possible \ac{lpwan} applications.
%\end{itemize}
%\item
%Explain problem at hand:
%\begin{itemize}
%\item
%What problem is your research trying to better understand or solve?

%Is the security of \ac{nb} as good as commonly claimed? Can \ac{lte} vulnerabilities be  applied to \ac{nb} networks?
%\item
%What is the scope of your study, general problem or something specific?

%Scope of this thesis is \ac{3gpp} Release~13, the release where \ac{nb} was specified first.
%\item
%What is your main claim or argument?

%\ac{nb} shares most vulnerabilities with \ac{lte}.
%As there are plenty of those, the security of \ac{nb} can be considered questionable.
%\end{itemize}
%\item
%Methods:
%\begin{itemize}
%\item
%Discuss own research including variables to approach

%Only on a conceptual level, not validated experimentally.
%Sometimes ambiguous which parts do or do not apply to \ac{nb} and thus could allow specific attacks.
%\item
%Evidence to support claim:

%\ac{lte} and \ac{nb} share considerable parts of their specifications.
%If a vulnerability is based on a specification section that is shared between \ac{lte} and \ac{nb} then there is no reason not to assume that \ac{nb} networks can be attacked therewith.
%\end{itemize}
%\item
%Results:

%It has been shown that most \ac{lte} vulnerabilities also apply to \ac{nb}.
%However, their impact on four exemplary \ac{iot} applications has been discussed and found to be only minor, mostly consisting of service disruptions to users.
%\item
%Es geht um Vernetzung, nicht um computing
%\item
%Abkürzungen einführen
%\item
%Was ist das besondere and LPWANs?
%\item
%Wozu braucht man LTE NB-IoT?
%\item
%Zusammenfassung der Ergebnisse
%\end{itemize}
