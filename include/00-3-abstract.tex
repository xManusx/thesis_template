%%%%%%%%%%%%%%%%%%%%%%%%%%%%%%%%%%%%%%%%%%%%%%%%%%%%%%%%%%%%%%%%%%%%%%%%%%%%%%%
%
% Abstract
% 
%%%%%%%%%%%%%%%%%%%%%%%%%%%%%%%%%%%%%%%%%%%%%%%%%%%%%%%%%%%%%%%%%%%%%%%%%%%%%%%

% Pseudo chapter
\chapter*{\ }
%\thispagestyle{plain}

%\pagebreak

\begin{otherlanguage}{ngerman}
	%\vspace*{0.10\textheight}
	\begin{center}
		\begin{large}
			\textbf{Zusammenfassung}
		\end{large}
	\end{center}
	\vspace{0.75em}

	\blindtext[3]
	%\vspace*{\fill}
\end{otherlanguage}
\acresetall

\cleardoublepage
%\thispagestyle{plain}
\chapter*{\ }
\acresetall

%\vspace*{0.10\textheight}
\begin{center}
	\begin{large}
		\textbf{Abstract}
	\end{large}
\end{center}
\vspace{0.75em}

\blindtext[3]
\acresetall

%\begin{itemize}
%\item
%Identify purpose:
%\begin{itemize}
%\item
%Why did I decide to study this topic?

%\Ac{iot} will be important topic in the future, expected growth is huge, vision behind it is fascinating
%\item
%How did I conduct my research?

%Examine both \ac{lte} and \ac{nb}, examine attacks on \ac{lte}, investigate their applicability to \ac{nb} on a theoretical level
%\item
%What did I find?

%Some attacks are based on features that are not supported in \ac{nb} and consequentially are not possible in \ac{nb}.
%Most attacks, however, are applicable to the new standard.
%\item
%Why is this research and findings important?

%Although the presented vulnerabilities have been shown to be mostly non-critical for the example applications, these results can serve as an example of \ac{nb} security research, that will be urgently needed in the future.
%As it is to be assumed that there will be a multitude of \ac{nb} applications in the future and the discovery of more attacks on \ac{lte}, respectively  \ac{nb}, can be expected, security considerations and research need to be a vital part of \ac{iot} application development.
%Thus, 
%\item
%Why should someone read the thesis entirely?

%It gives an overview of relevant parts of both \ac{lte} and \ac{nb}, necessary to understand the vulnerabilities in both standards and examine their impact on possible \ac{lpwan} applications.
%\end{itemize}
%\item
%Explain problem at hand:
%\begin{itemize}
%\item
%What problem is your research trying to better understand or solve?

%Is the security of \ac{nb} as good as commonly claimed? Can \ac{lte} vulnerabilities be  applied to \ac{nb} networks?
%\item
%What is the scope of your study, general problem or something specific?

%Scope of this thesis is \ac{3gpp} Release~13, the release where \ac{nb} was specified first.
%\item
%What is your main claim or argument?

%\ac{nb} shares most vulnerabilities with \ac{lte}.
%As there are plenty of those, the security of \ac{nb} can be considered questionable.
%\end{itemize}
%\item
%Methods:
%\begin{itemize}
%\item
%Discuss own research including variables to approach

%Only on a conceptual level, not validated experimentally.
%Sometimes ambiguous which parts do or do not apply to \ac{nb} and thus could allow specific attacks.
%\item
%Evidence to support claim:

%\ac{lte} and \ac{nb} share considerable parts of their specifications.
%If a vulnerability is based on a specification section that is shared between \ac{lte} and \ac{nb} then there is no reason not to assume that \ac{nb} networks can be attacked therewith.
%\end{itemize}
%\item
%Results:

%It has been shown that most \ac{lte} vulnerabilities also apply to \ac{nb}.
%However, their impact on four exemplary \ac{iot} applications has been discussed and found to be only minor, mostly consisting of service disruptions to users.
%\item
%Es geht um Vernetzung, nicht um computing
%\item
%Abkürzungen einführen
%\item
%Was ist das besondere and LPWANs?
%\item
%Wozu braucht man LTE NB-IoT?
%\item
%Zusammenfassung der Ergebnisse
%\end{itemize}
